%% KO_template.tex
%% V2015
%% by Jan Dvořák
%% based on
%% bare_conf.tex
%% V1.3
%% 2007/01/11
%% by Michael Shell

%%*************************************************************************
%% Legal Notice:
%% This code is offered as-is without any warranty either expressed or
%% implied; without even the implied warranty of MERCHANTABILITY or
%% FITNESS FOR A PARTICULAR PURPOSE! 
%% User assumes all risk.
%% In no event shall IEEE or any contributor to this code be liable for
%% any damages or losses, including, but not limited to, incidental,
%% consequential, or any other damages, resulting from the use or misuse
%% of any information contained here.
%%
%% All comments are the opinions of their respective authors and are not
%% necessarily endorsed by the IEEE.
%%
%% This work is distributed under the LaTeX Project Public License (LPPL)
%% ( http://www.latex-project.org/ ) version 1.3, and may be freely used,
%% distributed and modified. A copy of the LPPL, version 1.3, is included
%% in the base LaTeX documentation of all distributions of LaTeX released
%% 2003/12/01 or later.
%% Retain all contribution notices and credits.
%% ** Modified files should be clearly indicated as such, including  **
%% ** renaming them and changing author support contact information. **
%%
%% File list of work: IEEEtran.cls, IEEEtran_HOWTO.pdf, bare_adv.tex,
%%                    bare_conf.tex, bare_jrnl.tex, bare_jrnl_compsoc.tex
%%*************************************************************************


%%!!!!!!!!!!!!!!!!!!!!!!!!!!!!!!!!!!!!!!!!!!!!!!!!!!!!!!!!!!!!!!!!!!!!!!!!!
%%
%% Remove all itemize environments and replace them by your own text! 
%%
%%!!!!!!!!!!!!!!!!!!!!!!!!!!!!!!!!!!!!!!!!!!!!!!!!!!!!!!!!!!!!!!!!!!!!!!!!!



\documentclass[onecolumn, conference]{IEEEtran}

% Uncomment for semester project in czech language
%\usepackage[czech]{babel}

%Uncomment for setting up your encoding (according to *.tex encoding you use)
%\usepackage[latin2]{inputenc} %for windows users
%\usepackage[utf8]{inputenc} %for linux users

\newcommand{\range}[2][0]{Range: #1 to #2 page} 
\newcommand{\conciseItem}{\itemsep1pt \parskip0pt \parsep0pt}

\begin{document}
%
% paper title
% can use linebreaks \\ within to get better formatting as desired
\title{Comparison of Matrix Multiplication}


% author names and affiliationsb
% use a multiple column layout for up to three different
% affiliations
\author{\IEEEauthorblockN{Authors Name}
\IEEEauthorblockA{
Day and time of your parallel class\\
\textit{Study programme}\\
\textit{E-mail address}}
}

% make the title area
\maketitle
\section{Methodology}
\subsection{Warm-up}
\begin{itemize}
	\conciseItem
	\item Brief description of how the warm-up period is determined.
\end{itemize}
\subsection{Measurements}
\begin{itemize}
	\conciseItem
	\item Brief description of how the time performance is measured.
\end{itemize}
\subsection{Comparison}
\begin{itemize}
	\conciseItem
	\item Brief description of how the implementations are compared.
\end{itemize}
\section{Machine specification}
\begin{itemize}
	\conciseItem
	\item CPU, memory, OS, etc.
	\item JVM parameters if used.
\end{itemize}
\section{Results}
\subsection{Warm-up}
\begin{itemize}
	\conciseItem
	\item Results of the warm-up determination (with graphs) with a summary table.
\end{itemize}
\subsection{Measurements}
\begin{itemize}
	\conciseItem
	\item Results of the time measurements including a graph with displayed confidence intervals and a table.
\end{itemize}
\subsection{Comparison}
\begin{itemize}
	\conciseItem
	\item Results of the comparison of the implementations.
\end{itemize}


\section{Conclusion}
\begin{itemize}
	\conciseItem
	\item Summary of the results with a discussion
\end{itemize}

\end{document}
